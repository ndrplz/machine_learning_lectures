\documentclass[aspectratio=169]{beamer}
\mode<presentation>
{
  \usetheme{metropolis}      % or try Darmstadt, Madrid, Warsaw, ...
  \usecolortheme{default} % or try albatross, beaver, crane, ...
  \usefonttheme{structurebold}  % or try serif, structurebold, ...
  \setbeamercolor{background canvas}{bg=white}
  \setbeamertemplate{navigation symbols}{}
  \setbeamertemplate{bibliography item}{\insertbiblabel}
  %\setbeamertemplate{caption}[numbered]
} 
\usepackage[english]{babel}
\usepackage[utf8x]{inputenc}
\usepackage{listings}             % Include the listings-package
\hypersetup{
    colorlinks = true,
    linkcolor = {black},
    urlcolor = {blue}
}

\usepackage{animate}
\usepackage{listings}

\DeclareMathOperator*{\argmin}{arg\,min}

\title[Deep Learning and Temporal Data Processing]{Deep Learning and Temporal Data Processing}
\subtitle{Neural Networks in TensorFlow}
\institute{University of Modena and Reggio Emilia}
\author{Andrea Palazzi}
%\date{June 21th, 2017}

\def\thisframelogos{}

\newcommand{\framelogo}[1]{\def\thisframelogos{#1}}

\addtobeamertemplate{frametitle}{}{%
\begin{tikzpicture}[remember picture,overlay]
\node[anchor=north east] at (current page.north east) {%
    \foreach \img in \thisframelogos {%
        %\hspace{.5ex}%
        \includegraphics[height=3.5ex]{\img}%
    }%
};
\end{tikzpicture}}

\begin{document}

\framelogo{img/template/logo_unimore_white.png}

\bgroup
\renewcommand{\insertframenumber}{}
\begin{frame}[noframenumbering]
  \titlepage
\end{frame}
\egroup
\begin{frame}{Agenda}
  \tableofcontents
\end{frame}


%%%%%%%%%%%%%%%%%%%%%%%%%%%%%%%%%%%%%%%%%%%%%%%%%%%%%%%%%%%%%%%%%%
%%%%%%%%%%%%%%%%%%%%%%%%%%%%%%%%%%%%%%%%%%%%%%%%%%%%%%%%%%%%%%%%%%
%%%%%%%%%%%%%%%%%%%%%%%%%%%%%%%%%%%%%%%%%%%%%%%%%%%%%%%%%%%%%%%%%%

\section{MNIST}

\begin{frame}{MNIST Dataset}
\textbf{MNIST\cite{lecun1998mnist} is a database of handwritten digits} consisting of 60K training images and 10K testing images. All digits have been centered in 28x28 grayscale images.
\begin{figure}
\begin{tabular}{c}
\includegraphics[width=0.9\textwidth]{img/tf/mnist.png}
\end{tabular}
\end{figure}
\end{frame}


\begin{frame}[fragile]{MNIST Dataset}
Due to its simplicity (in 2017!) the MNIST dataset is often considered to be the "Hello World!" in the Machine Learning framework.\\
\vspace{0.5cm}
In \texttt{lab\_utils.py} I already implemented a function \texttt{get\_mnist\_data} that handles the download and loading of the dataset.\\
\begin{verbatim}
    # Load MNIST data
    mnist = get_mnist_data('/tmp/mnist', verbose=True)
\end{verbatim}
\end{frame}


%%%%%%%%%%%%%%%%%%%%%%%%%%%%%%%%%%%%%%%%%%%%%%%%%%%%%%%%%%%%%%%%%%
%%%%%%%%%%%%%%%%%%%%%%%%%%%%%%%%%%%%%%%%%%%%%%%%%%%%%%%%%%%%%%%%%%
%%%%%%%%%%%%%%%%%%%%%%%%%%%%%%%%%%%%%%%%%%%%%%%%%%%%%%%%%%%%%%%%%%

\section{Goal}

%%%%%%%%%%%%%%%%%%%%%%%%%%%%%%%%%%%%%%%%%%%%%%%%%%%%%%%%%%%%%%%%%%

\begin{frame}{Goal}
The \textbf{goal} of this practice is to \textbf{implement a fully-connected neural network to perform 10-class classification} on the MNIST dataset.
\end{frame}

%%%%%%%%%%%%%%%%%%%%%%%%%%%%%%%%%%%%%%%%%%%%%%%%%%%%%%%%%%%%%%%%%%

\begin{frame}{Outline}
Some advices:
\begin{enumerate}
\item Take your time to explore the dataset
\item Start small and make the whole training pipeline working
\item First implement a single hidden layer neural network
\item Implement a bigger network: how does this affects performance?
\end{enumerate}
\end{frame}


%%%%%%%%%%%%%%%%%%%%%%%%%%%%%%%%%%%%%%%%%%%%%%%%%%%%%%%%%%%%%%%%%%
%%%%%%%%%%%%%%%%%%%%%%%%%%%%%%%%%%%%%%%%%%%%%%%%%%%%%%%%%%%%%%%%%%
%%%%%%%%%%%%%%%%%%%%%%%%%%%%%%%%%%%%%%%%%%%%%%%%%%%%%%%%%%%%%%%%%%

\section{References}

\begin{frame}[t, allowframebreaks]
\frametitle{References}
\bibliographystyle{abbrv}
\bibliography{bibliography}
\end{frame}

\end{document}